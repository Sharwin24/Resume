%%%%%%%%%%%%%%%%%
% This is an sample CV template created using altacv.cls
% (v1.3, 10 May 2020) written by LianTze Lim (liantze@gmail.com). Now compiles with pdfLaTeX, XeLaTeX and LuaLaTeX.
% This fork/modified version has been made by Nicolás Omar González Passerino (nicolas.passerino@gmail.com, 15 Oct 2020)
%
%% It may be distributed and/or modified under the
%% conditions of the LaTeX Project Public License, either version 1.3
%% of this license or (at your option) any later version.
%% The latest version of this license is in
%%    http://www.latex-project.org/lppl.txt
%% and version 1.3 or later is part of all distributions of LaTeX
%% version 2003/12/01 or later.
%%%%%%%%%%%%%%%%

%% If you need to pass whatever options to xcolor
\PassOptionsToPackage{dvipsnames}{xcolor}

%% If you are using \orcid or academicons
%% icons, make sure you have the academicons
%% option here, and compile with XeLaTeX
%% or LuaLaTeX.
% \documentclass[10pt,a4paper,academicons]{altacv}

%% Use the "normalphoto" option if you want a normal photo instead of cropped to a circle
% \documentclass[10pt,a4paper,normalphoto]{altacv}

\documentclass[10pt,a4paper,ragged2e,withhyper]{altacv}

%% AltaCV uses the fontawesome5 and academicons fonts
%% and packages.
%% See http://texdoc.net/pkg/fontawesome5 and http://texdoc.net/pkg/academicons for full list of symbols. You MUST compile with XeLaTeX or LuaLaTeX if you want to use academicons.

% Change the page layout if you need to
% \geometry{left=0.85cm,right=0.85cm,top=0.85cm,bottom=0.85cm,columnsep=0.75cm}
\geometry{left=0.825cm,right=0.825cm,top=0.825cm,bottom=0.825cm,columnsep=0.70cm}

% The paracol package lets you typeset columns of text in parallel
\usepackage{paracol}
\usepackage{hyperref}

% Change the font if you want to, depending on whether
% you're using pdflatex or xelatex/lualatex
\ifxetexorluatex
  % If using xelatex or lualatex:
  \setmainfont{Roboto Slab}
  \setsansfont{Lato}
  \renewcommand{\familydefault}{\sfdefault}
\else
  % If using pdflatex:
  \usepackage[rm]{roboto}
  \usepackage[defaultsans]{lato}
  % \usepackage{sourcesanspro}
  \renewcommand{\familydefault}{\sfdefault}
\fi

\definecolor{SlateGrey}{HTML}{2E2E2E}
\definecolor{LightGrey}{HTML}{666666}
\definecolor{PrimaryColor}{HTML}{0A6069} % Dark Blue: 001F5A
\definecolor{SecondaryColor}{HTML}{13737C} % Lighter Blue: 0039AC
\definecolor{ThirdColor}{HTML}{D22B2B} % Orange: F3890B
\definecolor{Text}{HTML}{2E2E2E}
\definecolor{BackgroundColor}{HTML}{E2E2E2}
\colorlet{name}{Text}
\colorlet{tagline}{Text}
\colorlet{heading}{Text}
\colorlet{headingrule}{Text}
\colorlet{subheading}{Text}
\colorlet{accent}{Text}
\colorlet{emphasis}{Text}
\colorlet{body}{Text}
\pagecolor{BackgroundColor}

\usepackage{hyperref}
\hypersetup{
% pdfborder = {0,0,0},
breaklinks = true,
bookmarksnumbered = true,
pdftitle = {SharwinPatil Resume},
pdfauthor = {Sharwin Patil},
pdfnewwindow = true,      % links in new window
pdfstartview = {FitH},    % initial view
}

% Change some fonts, if necessary
\renewcommand{\namefont}{\LARGE\rmfamily\bfseries}
\renewcommand{\personalinfofont}{\small\bfseries}
\renewcommand{\cvsectionfont}{\Large\rmfamily\bfseries}
\renewcommand{\cvsubsectionfont}{\normalsize\bfseries}


% Change the bullets for itemize and rating marker
% for \cvskill if you want to
\renewcommand{\itemmarker}{{\small\textbullet}}
\renewcommand{\ratingmarker}{\faCircle}

%% sample.bib contains your publications
%% \addbibresource{sample.bib}

% Roman Numeral command
\newcommand{\rom}[1]{\uppercase\expandafter{\romannumeral #1\relax}}

\begin{document}
\name{Sharwin Patil}
\tagline{}
%% You can add multiple photos on the left or right
% \photoL{4cm}{closePhoto}
\personalinfo{
    \email{sharwinpatil@u.northwestern.edu}
    \phone{925-389-8466}
    \homepage{sharwinpatil.info}
    \location{Chicago, IL}
}
\makecvheader

% Make fonts of itemize environments slightly smaller
\AtBeginEnvironment{itemize}{\small}

%% Set the left/right column width ratio to 6:4. (0.75)
\columnratio{0.68} % 0.65 - 0.75
    \vspace{-1.75em}
    % ----- EDUCATION -----
    \cvsection{Education}
    \cvevent{\textbf{Northwestern University}}{| \textbf{Master of Science in Robotics}}{\textit{Expected Graduation: 12/2025}}{\textit{Chicago, IL}}
    \cvevent{\textbf{Northeastern University}}{| \textbf{B.S. in Computer Engineering \& Computer Science, Minor in Robotics}}{\textit{Graduated: 05/2024}}{\textit{Boston, MA}}
    % ----- EDUCATION -----
    \vspace{-1.6em}
    \cvsection{Skills}
    \begin{itemize}
        \item[] \textbf{Robotics:} ROS/ROS2, Nav2, MoveIt, RViz2, Gazebo, OpenCV, Robot Kinematics, Dynamics, \& Manipulation, Kalman Filtering
        \item[] \textbf{Software:} Embedded C, C++, Python, C\#, .NET, Java, Matlab, Linux, Git, Unity, Docker, SolidWorks (CSWA)
        \item[] \textbf{Machine Learning:} Computer Vision, PyTorch, CNNs, Reinforcement Learning (DQN, PPO), Mujoco
    \end{itemize}
    \vspace{-1.5em}
    % ----- EXPERIENCE -----
    \cvsection{Experience}
    \cvevent{\textbf{Locus Robotics}}{| \textbf{Planning \& Controls Intern}}{\textit{06/2025 -- 09/2025}}{\textit{Remote}}
    \begin{itemize}
        \item Optimizing fleet behavior with new ROS services bridging navigation stack (C++) and the robot behavior stack (Python)
        \item Performance-tested ROS services on large-scale maps with hundreds of data points, ensuring scalability for fleets of hundreds of robots
    \end{itemize}
    \cvevent{\textbf{GreenSight}}{| \textbf{UAV Robotics Engineer Co--op}}{\textit{06/2023 -- 12/2023}}{\textit{Boston, MA}}
    \begin{itemize}
        \item Developed RTOS firmware for communications between a swarm of drones and GCS over LoRa for collecting real-time weather data
        \item Implemented a Hardware-Abstraction-Layer (HAL) in C for the ESP32 platform to interface with a custom LoRa chipset
        \item Created a custom SPI driver for the ESP32 to connect to peripherals for the drone's data collection
    \end{itemize}
    \cvevent{\textbf{Fulfil Solutions}}{| \textbf{Robotics Software Controls Co--op}}{\textit{07/2022 -- 12/2022}}{\textit{Redwood City, CA}}
    \begin{itemize}
        \item Developed planning code in C\# for high-level behavior and task assignment for up to 50 robots in a warehouse automation setting
        \item Composed data fetching functions to bridge C\# sequencing code to MongoDB
        \item Optimized AGV planning and curated heuristics for maintaining the factory's health while improving performance
        \item Deployed factory-wide alerts and notifications for operators to react with relevant safety measures
    \end{itemize}
    \cvevent{\textbf{Doble Engineering}}{| \textbf{Software Engineering Co--op}}{\textit{07/2021 -- 12/2021}}{\textit{Marlborough, MA}}
    \begin{itemize}
        \item Developed an external data persistence mechanism in C\# running on the .NET framework for various Doble software products
        \item Designed and deployed a firmware installation wizard using Windows Presentation Foundation (WPF) for Doble instruments
    \end{itemize}
    % ----- EXPERIENCE -----
    \vspace{-1.5em}
    % ----- PROJECTS -----
    \cvsection{Projects}
    \cvevent{\textbf{Open Sourced ROS2 Delta Robot}}{}{\textit{01/2025 -- 03/2025}}{}
    \begin{itemize}
        \item Designed and fabricated a 3-DOF delta robot using 3D printed parts and Dynamixel motors
        \item Derived Jacobian for planning and executing both position and velocity trajectories
        \item Authored a configurable ROS 2 package to interface with any delta robot and provide motion planning and sensor telemetry
    \end{itemize}
    \cvevent{\textbf{Toasting Bread with Franka Robot Arm}}{}{\textit{11/2024 -- 12/2024}}{}
    \begin{itemize}
        \item Implemented a ROS2 package to interface with the MoveIt API for sending motion requests to the Franka robot arm
        \item Utilized an Intel Realsense camera to identify april tag markers for the robot to interact with the scene
        \item Collaborated within a group of students to sequence the camera and robot with the scene elements to autonomously toast bread
    \end{itemize}
    \cvevent{\textbf{KUKA Mobile Manipulator Pick and Place Simulation}}{}{\textit{11/2024 -- 12/2024}}{}
    \begin{itemize}
        \item Implemented a task-space feed-forward controller for the KUKA YouBot in Python for pick and place tasks
        \item Derived Jacobian for omni-directional mobile base and robotic arm to generate velocity trajectories for robot arm and the mobile base
    \end{itemize}
    % \cvevent{\textbf{Automated Poker Table}}{}{\textit{01/2023 -- 04/2023}}{}
    % \begin{itemize}
    %     \item Designed an automated card shuffler and dealer with stepper motors, 3D printed parts, and sensors for detecting cards
    %     \item Developed firmware for I2C and serial communications between STM32 microcontrollers and a Raspberry Pi
    %     \item Trained a Convolutional Neural Network to classify cards for game management and software shuffling
    %     \item Earned first place for Northeastern's ECE Capstone 2023
    % \end{itemize}
    \cvevent{\textbf{Kalman Filter for 6-Wheel Mars Rover}}{}{\textit{08/2025 -- Present}}{}
    \begin{itemize}
        \item Derived and implemented an Extended Kalman filter in C++ for state estimation, fusing data from an IMU, wheel encoders, and a LiDaR
        \item Simulated the rover's motion system in Python using NumPy and Matplotlib to tune process and measurement noise covariances
        \item Architected a ROS2 Package to interface with sensors and handle real-time data processing
    \end{itemize}
    % ----- PROJECTS -----
\end{document}